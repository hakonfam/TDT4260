% Evaluation criterion:
%- Language and use of figures
%- Clarity of the problem statement
%- Overall document structure
%- Depth of understanding for the field of computer architecture
%- Depth of understanding of the investigated problem

\section{Methodology}
\label{sec:methodology}

To discover what the effect of using different prefetching strategies
for different applications are, we used a modified version of the M5
simulator, version 2.0\cite{M5} as a testing platform. This simulator
is a software implementation loosely based on the Alpha 21264
processor\cite{Alpha}. The version used was provided by the course
staff, and included an interface to the L2 cache prefetching
functionality of the simulator. By using this interface, we could
integrate our prefetchers into the processor simulator, such that the
resulting binaries would be a simulator of the processor with a
specific prefetcher algorithm in control of prefetching for the L2 cache.

For each of the simulators built this way, we executed a suite of
programs from the SPEC CPU2000 benchmark. The programs used are listed
in table~\ref{tab:benchmarks}. The simulator software automatically
tracked key statistics during the program execution, such as cache
misses, prefetches issued, and clock cycles used for running the
program. By comparing the number of cycles used when executing the
program with our prefetchers included in the simulator to the number
of cycles used without a prefetcher, we could calculate the speedup
gains due to the prefetcher for the different programs.

\begin{table}[htbp]
  \centering
  \begin{tabular}{|c|}
    \hline
    {\bf Benchmarks} \\ \hline
    ammp \\ \hline
    applu \\ \hline
    apsi \\ \hline
    art110 \\ \hline
    art470 \\ \hline
    bzip2\_graphic \\ \hline
    bzip2\_program \\ \hline
    bzip2\_source \\ \hline
    galgel \\ \hline
    swim \\ \hline
    twolf \\ \hline
    wupwise \\ \hline
  \end{tabular}
  \caption{The benchmarks used to evaluate the prefetcher efficiency.}
  \label{tab:benchmarks}
\end{table}

% Give prefetcher configurations
When implementing our prefetchers, there were a number of
implementation parameters which had to be set (e.g. the size of the
data structures used in the algorithm, the number of addresses to
prefetch). To get fair comparisons between the different strategies,
we wanted to find the settings which yielded the best results for each
prefetcher. To make the prefetchers more realistic, we were only
allowed to use 8 KiB of storage internally in our prefetcher. This set
the bounds for our prefetcher configuration space. 

Our strategy for finding the optimal settings for the given prefetcher
was to set them initially to more or less arbitrary values within the
permissible range, and attempt to locate at least a local maximum by
varying the parameters in turn. Due to the enormous possible set of
values, only a few were tried for each degree of freedom---these are
listed in \autoref{tab:stridesettings}, \autoref{tab:ghbsettings} and
\autoref{tab:smssettings}. The
combination of parameters which yielded the highest average speedup
are marked in bold.

\begin{table}[htbp]
  \centering
  \begin{tabular}{|c|c|c|}
    \hline
    \textbf{Prefetch Degree} & \textbf{Consecutive strides} \\ \hline
    10 & 3 \\ \hline
    \textbf{15} & \textbf{4} \\ \hline
    15 & 5 \\ \hline % TODO: Not all these settings have been attempted yet...
  \end{tabular}
  \caption{The various parameters attempted for the ABSP. The prefetch degree is the number of consecutive blocks prefetched, the consecutive strides is how many consecutive constant step memory misses must occur before a prefetch is issued.}
  \label{tab:stridesettings}
\end{table}

\begin{table}[htbp]
  \centering
  \begin{tabular}{|c|c|c|}
    \hline
    \textbf{Table Size} & \textbf{Prefetch Degree} \\ \hline
    300 & 3 \\ \hline
    \textbf{350} & 5 \\ \hline
    400 & \textbf{10} \\ \hline
        & 15 \\ \hline
  \end{tabular}
  \caption{The various parameters attempted for the GHB/PCDC prefetcher. The table size is the number of entries in both the GHB and the index table---unequal sizes were not tested. The prefetch degree is the number of blocks to issue prefetches for, once you find a previous history match.}
  \label{tab:ghbsettings}
\end{table}

\begin{table}[htbp]
  \centering
  \begin{tabular}{|c|c|c|c|c|}
    \hline
    \textbf{Region (log2)} & \textbf{History Table} & \textbf{Filter Table} & \textbf{Accumulation Table} \\ \hline
    4  & 128 & 16 & 16 \\ \hline
    8  & \textbf{256} & 32&  \textbf{32}\\ \hline
    \textbf{11} &     & \textbf{64}&  64\\ \hline
  \end{tabular}
  \caption{The various parameters attempted for the SMS prefetcher.}
  \label{tab:smssettings}
\end{table}



% \begin{table}[htbp]
%   \centering
%   \begin{tabular}{|c|c|}
%     \hline
%     \textbf{Prefetcher} & \textbf{Settings} \\ \hline
%     Stride & Wait for 4 consecutive strides, prefetch 15 blocks \\ \hline
%     GHB/PCDC & 350 entries in GHB and index table, prefetch 10 blocks  \\ \hline
%     SMS & \\ \hline
%     Markov & \\ \hline
%   \end{tabular}
%   \caption{The settings yielding optimal overall result for each prefetcher.}
%   \label{tab:settings}
% \end{table}


%Having  (... cannot remember what this was supposed to start :X)XS

% (should M5 description be in introduction?)

% (Should we say that they tests were run on a cluster?)  {\bf
% (references, citations)}

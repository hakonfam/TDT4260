% Evaluation criterion:
%- Language and use of figures
%- Clarity of the problem statement
%- Overall document structure
%- Depth of understanding for the field of computer architecture
%- Depth of understanding of the investigated problem

\section{Methodology}
\label{sec:methodology}

To discover what the effect of using different prefetching strategies
for different applications are, we used the M5 simulator. This
simulator is a software implementation loosely based on the Alpha
21264 processor \cite{TDT4260 Staff} (should M5 description be in
introduction?). In order to test our prefetcher implementations, each
of them were implemented as a submodule of M5, and built into the
final simulator binary. For each of the simulators built this way, we
executed a suite of programs from the SPEC CPU2000 benchmark. The
simulator software automatically tracked key statistics during the
program execution, such as cache misses, prefetches issued, and clock
cycles used for running the program. By comparing the number of cycles
used when executing the program with our prefetchers included in the
simulator to the number of cycles used without a prefetcher, we could
calculate the speedup gains due to the prefetcher for the different
programs. (Should we say that they tests were run on a cluster?)
{\bf (references, citations)}


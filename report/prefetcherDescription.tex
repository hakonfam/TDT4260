% Evaluation criterion:
%- Language and use of figures
%- Clarity of the problem statement
%- Overall document structure
%- Depth of understanding for the field of computer architecture
%- Depth of understanding of the investigated problem

\section{Prefetcher Description}
\label{sec:prefetcherDescription}

In our paper, we will consider four different prefetching strategies,
each taken from a separate category of prefetchers identified in
Grannas' paper (reference needed).

\subsection{Sequential Prefetcher}
\label{sec:sequentialPrefetcher}
The sequential prefetcher ... 
{\bf (references, citations)}

\subsection{Global History Buffer Prefetcher}
\label{sec:ghbPcdcPrefetcher}
The global history buffer prefetcher is an instruction based
prefetching scheme. In this schem, a datastrcture called Global
History Buffer (abbreviated GHB) is maintained. This datastructure is
a table which containsr  ... We implement a variety called GHB/PCDC (Global History
Buffer, Program Counter based with Delta Correlation).  {\bf
  (references, citations)}

\subsection{Markov Prefetcher}
\label{sec:markovPrefetcher}
The Markov prefetcher is an address based prefetcher,...
{\bf (references, citations)}

\subsection{Spatial Memory Streaming Prefetcher}
\label{sec:smsPrefetcher}
The Spatial Memory Streaming (SMS) Prefetcher is a spatial locality
prefetcher...  {\bf (references, citations)}

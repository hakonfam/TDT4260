\paragraph{Combining Prefetchers}
\label{par:prefcombo}
Although the GHB prefetcher has the best performance on most
applications, it is outperformed by the ABSP prefetcher on the \emph{galgel}
and \emph{apsi} benchmarks. With a more varied workload and with other
prefetcher algorithms, the diversity in efficiency might have been
even greater. To try to get the best of all worlds, a natural line of
thought would be to attempt to combine two or more of the
prefetchers. This could be done in two ways. One way would be to try
and merge the two algorithms into a single prefetching
algorithm. Another way would be to create a controlling unit utilizing
heuristics to choose which of the prefetchers should be used at a
given time.

The first such strategy is especially applicable to the stride and GHB
prefetchers. The GHB is in essence almost a stride prefetcher, since a
constant delta stride will also be recognized. However, as there would
only be two more items left in the delta table after the point of
matching, only two addresses will be prefetched. According to
\autoref{tab:stridesettings}, this is most likely fewer addresses than
would be optimal. Therefore, one could try to recognize this special
case by checking when the final four elements of the delta table are
identical, and prefetch a larger set of blocks the constant stride
apart.

When running the GHB prefetcher with this change included, the optimal
configuration is a table size of $350$ entries and a prefetch degree
of $5$. With this configuration, the average speedup is $1.058$. The
performance on \emph{apsi} is increased to $1.051$, surpassing the ABSP
prefetcher. On the \emph{galgel} benchmark, the performance is increased to
$1.029$ up from $1.016$. This is a bit closer to the performance of
the ABSP prefetcher, which has a speedup of $1.058$, but is still not
quite as good. We were as such unable to easily extend the GHB
prefetcher to provide as good performance as the ABSP prefetcher in
its dominant area.

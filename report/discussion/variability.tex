
\paragraph{Analysis of Prefetcher Performance}
\label{par:varprefperf}

%- Introduction?


%Should the data in here be in result instead? Percentages etc.
From these results, there are a number of interesting observations one
might make. The first item of interest relates to the average
prefetcher speedups. As one might expect, it is the most complex,
general purpose prefetcher, namely the GHB/PCDC prefetcher, which
yields the highest average speedup. With an average speedup of
$1.054$, it outperforms ABSP by $54.3\%$, SMS by $500\%$ and Markov by
$some\%$. As it attempts to improve upon weaknesses in earlier methods
(such as the stride directed and Markov prefetchers~\cite{Nesbit}),
our results can be taken as an indication of its success. It therefore
shows that there may be merit in creating a prefetcher which handles
nontrivial memory access patterns, which, although not particularly
surprising, is nevertheless reassuring for the computer scientist
community.% TODO: Remove GPP?

A second point of interest is the relatively poor performance of the
SMS prefetcher. Although it is arguably the most complex of all the
prefetcher implementations, it yields an average speedup which is only
$25\%$ of our simplest prefetcher, the ABSP. This is explainable by
the nature of the benchmark programs we ran the prefetchers with. As
explained in \autoref{sec:smsPrefetcher}, the SMS prefethcer is
designed to increase the performance of applications such as operating
systems and databases. However, there are no such programs included in
the SPEC CPU2000 suite, for which such I/O-intensive programs would
not be suitable~\cite{SPECFAQ}. As the programs exhibit a lower
degree of the memory access patterns the SMS prefetcher attempts to
recognize, it will be unable to be of much use. As can be seen in
\autoref{tab:numPrefetches}, the prefetcher is not completely
inactive, and is able to recognize some patterns. However, its
activity level is in general lower than the other prefetchers'. Thus,
using the SMS prefetcher in a domain it was not designed for is not
very beneficial.

%The more complex prefetchers perform better on average. Specifically,
%the GHB prefetcher has the highest performance. However, the
%difference is not very large except for a few benchmarks. These
%exceptions are most likely occuring because the benchmark programs
%have a memory access pattern which corresponds well with the
%prefetcher which excels.
%Blowup ideas:
%       - Define complex - LoC table. Also, SMS is most complex, but worst?
%       - GHB highest by how much? Specify the avg values, hard to read on the graph.
%n       - Aren't the differences very large? Specify percentages to justify such a claim
%       - The last sentence needs clarification

%For instance, on the benchmark ammp the GHB prefetcher is able to
%achieve many times the speedup of the other prefetchers. This program
%operates on the 

%This point is a bit similar to our first paragraph, but is meant to
%make concrete the point that it is the extra access pattern
%recognition which makes it better whereas the first paragraph is
%stated in more general terms of expectations. Are both necessary,
%though?
When studying the cause of the superior performance of the GHB
prefetcher in further detail, it is evident that one benchmark in
particular, namely ammp, makes this prefetcher pull ahead of the other
algorithms. The geometric mean of the speedups when disregarding ammp
is $1.045$, whereas the absp prefetcher, for instance, rises to
$1.39$, reducing the difference in performance considerably.

About ammp:
- The one which drags avg of ghb up
- Highest potential for speedup, cf \cite{Nesbit}.
- Thoughts:
  * Can be very beneficial to recognize more patterns, as it is in this case.
  * Not so much for workloads were the pattern is not present or
    prefetching does not have much impact on performance.
  * A bit similar to SMS, but since it is more general it stands its ground. (?)

%Explanation of ammp: refer to GHB paper stating that this has the
%highest potential for speedup? Perhaps the paper also states the
%memory access pattern in ammp?

%Paragraph should perhaps be in uncertainties?
An interesting remark is that even though the ABSP is a simplication
of the prefetcher that GHB is an extension of, it has similar
performance gains on several of the benchmark programs. Discuss
whether this might make it an attractive alternative, given its
relative simplicity? Or is this in uncertainties?

Another factor that is not taken into consideration is that the two address based prefetchers (ABSP and markov) would occupy less critical CPU surface, as it would not require having access to the program counter. The program counter based prefetchers would need to be implemented close enough to the CPU to have access to the program counter, or would need to have the program counter passed on to them. Both of theses solutions would affect the overall performance enhancement of the prefetchers. 


%Paragraph should perhaps be in combining?
One benchmark on which the GHB prefetcher was outperformed was the
galgel benchmark, on which the stride prefetcher got an average
speedup almost five times as high.


%  - Are there any variance as to which programs they yield good speedup
%  on?  

%  According to the results in \autoref{fig:initResults}.  The more
%  complex ones are usually better overall, however there are some
%  programs which 

% - How would combining them work?


\paragraph{Analysis of Prefetcher Performance}
\label{par:varprefperf}

%- Introduction?


%Should the data in here be in result instead? Percentages etc.
From these results, there are a number of interesting
observations. The first item of interest relates to the average
prefetcher speedups. As one might expect, it is the most complex,
general purpose prefetcher, namely the GHB/PCDC prefetcher, which
yields the highest average speedup. With an average speedup of
$1.054$, it outperforms ABSP by $54.3\%$, SMS by $500\%$ and Markov by
$1700\%$. As it attempts to improve upon weaknesses in earlier methods
(such as the stride directed and Markov prefetchers~\cite{Nesbit}),
our results can be taken as an indication of its success. It therefore
shows that there may be merit in creating a prefetcher which handles
nontrivial memory access patterns, which, although not particularly
surprising, is nevertheless reassuring for the computer scientist
community.% TODO: Remove GPP?

A second point of interest is the relatively poor performance of the
SMS prefetcher. Although it is one of the more complex of the four
prefetcher implementations, it yields an average speedup which is only
$25\%$ of our simplest prefetcher, the ABSP. This is explainable by
the nature of the benchmark programs we ran the prefetchers with. As
explained in \autoref{sec:smsPrefetcher}, the SMS prefethcer is
designed to increase the performance of applications such as operating
systems and databases. However in the SPEC CPU2000 suite, there are no
I/O intensive applications~\cite{SPECFAQ}, which are more likely to
have been tuned to have high spatial locality on disk block
granularity. As the programs exhibit a lower degree of the memory
access patterns the SMS prefetcher attempts to recognize, it will be
unable to be of much use. As can be seen in
\autoref{tab:numPrefetches}, the prefetcher is not completely
inactive, and is able to recognize some patterns. However, its
activity level is in general lower than the other prefetchers'. Thus,
using the SMS prefetcher in a domain it was not designed for is not
very beneficial. In addition, according to~\cite{SMS} the optimal
configuration has a memory usage requirement on the order of hundreds
of kilobytes, which was outside our parameter search space. This could
also account for some of the poor performance.

The Markov prefetcher also performed poorly on average. One of the
reasons for this is that it has a very high memory requirement (on the
order of megabytes, according to \cite{Nesbit}). When being restricted
by a maximum of 8KiB the Markov prefetcher is unable to contain a
large enough graph to be able to perform accurate predictions. Since
the Markov prefetcher only is able to prefetch addresses which it has
already seen, it is unable to yield a high speedup for applications
with a low rate of repeating memory addresses.

It is interesting to note, however, that the Markov prefetcher is the
one with the highest speedup on any single benchmark. Although it has
the worst performance on ten out of twelve benchmarks, it apparently
is the best match for the memory access behavior of \emph{ammp}. If a
workload consisted only of this kind of workload, then the Markov
prefetcher would arguably be a valid choice.

When studying the cause of the superior performance of the GHB
prefetcher in further detail, it is evident that one benchmark in
particular, namely \emph{ammp}, makes this prefetcher pull ahead of
the other algorithms. The harmonic mean of the speedups when
disregarding \emph{ammp} is $1.045$, whereas the ABSP prefetcher, for
instance, rises to $1.038$, reducing the difference in performance
considerably. According to table 2 in \cite{Nesbit}, \emph{ammp} is
the application which has the greatest potential for speedup from
prefetching data, with the possibility of $815\%$ improvement in
execution time. Thus, being able to leverage this potential gives a
significant competitive edge. This also shows that using a prefetcher
that can match several access patterns will have a larger chance of
matching the access patterns of the applications, and that this can be
significant for the average performance of the prefetcher. In contrast
to the SMS prefetcher, the performance is still good even when
disregarding the best input case. This is a consequence of its generic
nature, attempting to match both normal and more rare access patterns.

An interesting remark is that even though the ABSP is a simplification
of the prefetcher that GHB is an extension of, it has similar
performance gains on several of the benchmark programs. As previously
noted, when disregarding the \emph{ammp} benchmark, its results is
only $13.6\%$ worse than that of GHB. Considering the relative
simplicity and low memory requirements of the ABSP prefetcher this
difference might be too small to justify the choice of a prefetcher as
complex as the GHB. This shows the value of investigating the context
in which the prefetcher will be used.

To be specific, one factor that is not taken into consideration is
that the ABSP and Markov prefetcher most likely would occupy less
critical CPU surface, as they would not require having access to the
program counter. The two prefetchers relying on the program counter
(SMS and GHB) would need to be implemented close enough to the CPU to
have access to the program counter, or would need to have the program
counter passed on to them. Both of these solutions are likely to have
a negative impact on the performance of the prefetchers.



\paragraph{Variability of Prefetcher Performance}
\label{par:varprefperf}

%- Introduction?


%Should the data in here be in result instead? Percentages etc.
From these results, there are a number of interesting observations one
might make. The first item of interest relates to the average
prefetcher speedups. As one might expect, it is the most complex,
general purpose prefetcher, namely the GHB/PCDC prefetcher, which
yields the highest average speedup. With an average speedup of $1.54$,
it outperforms ABSP by $54.3\%$, SMS by $500\%$ and Markov by
$some\%$. As it attempts to improve upon weaknesses in earlier methods
as the stride directed and Markov prefetchers~\cite{Nesbit}, our
results can be taken as an indication of its success. It therefore
shows that there may be merit in creating a prefetcher which handles
nontrivial memory access patterns, which, although not particularly
surprising, is nevertheless reassuring for the computer scientist
community.% TODO: Remove GPP?

A second point of interest is the relatively poor performance of the
SMS prefetcher. Although it is arguably the most complex of all the
prefetcher implementations, it yields an average speedup which is only
$25\%$ of our simplest prefetcher, the ABSP. This is explainable by
the nature of the benchmark programs we ran the prefetchers with. As
explained in \autoref{sec:smsPrefetcher}, the SMS prefethcer is
designed to increase the performance of applications such as operating
systems and databases. However, there are no such programs included in
the SPEC CPU2000 suite, for which such I/O-intensive programs would
not be suitable~\cite{SPECFAQ}. As the programs exhibits a lower
degree of the memory access patterns the SMS prefetcher attempts to
recognize, it will be unable to be of much use. As can be seen in
\autoref{tab:numPrefetches}, the prefetcher is not completely
inactive, and is able to recognize some patterns. However, its
activity level is in general lower than the other prefetchers'. Thus,
using the SMS prefetcher in a domain it was not designed for is not
very beneficial.

%The more complex prefetchers perform better on average. Specifically,
%the GHB prefetcher has the highest performance. However, the
%difference is not very large except for a few benchmarks. These
%exceptions are most likely occuring because the benchmark programs
%have a memory access pattern which corresponds well with the
%prefetcher which excels.
%Blowup ideas:
%       - Define complex - LoC table. Also, SMS is most complex, but worst?
%       - GHB highest by how much? Specify the avg values, hard to read on the graph.
%n       - Aren't the differences very large? Specify percentages to justify such a claim
%       - The last sentence needs clarification

%For instance, on the benchmark ammp the GHB prefetcher is able to
%achieve many times the speedup of the other prefetchers. This program
%operates on the 

 Explanation of ammp: refer to GHB paper stating that this has the
 highest potential for speedup? Perhaps the paper also states the
 memory access pattern in ammp?

One benchmark on which the GHB prefetcher was outperformed was the
galgel benchmark, on which the stride prefetcher got an average
speedup almost five times as high.

An interesting remark is that even though the ABSP is a simplication
of the prefetcher that GHB is an extension of, it has similar
performance gains on several of the benchmark programs.


%  - Are there any variance as to which programs they yield good speedup
%  on?  

%  According to the results in \autoref{fig:initResults}.  The more
%  complex ones are usually better overall, however there are some
%  programs which 

% - How would combining them work?

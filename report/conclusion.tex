% Evaluation criterion:
%- Language and use of figures
%- Clarity of the problem statement
%- Overall document structure
%- Depth of understanding for the field of computer architecture
%- Depth of understanding of the investigated problem

\section{Conclusion}
\label{sec:conclusion}

In this paper, we compared a number of prefetching algorithms run on
several different applications in order to compare to what degree they
sped up program execution on average. We also investigated whether
there was any difference with regards to which applications the
different prefetchers performed well on, or if there was a strategy
which clearly excelled.

When analyzing the overall results, it is clear that the GHB/PCDC prefetcher yields the highest speedup on average. However, this was mainly because of its superior performance on one particular application. On the other hand, the special purpose SMS prefetcher was used outside of its intended domain, and got poor results.  

Overall, it is clear that when choosing a prefetcher in a design, the context of its use would be critical. This applies not only to the type of applications it will operate on, but also to the hardware environment. In a system with a small amount of memory available, the ABSP would be a better choice than the SMS prefetcher, whereas a database system would get the best performance gain using an SMS prefetcher \cite{SMS}. 

All in all, we have seen that prefetchers vary vastly in complexity, and in performance gains. Different types of applications get varying speedups from the same prefetcher. For some specific tasks (HPC), creating a tailor-made prefetcher for one kind of application might be worth the effort, as opposed to using a general one.

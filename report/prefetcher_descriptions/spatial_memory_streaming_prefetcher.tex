\subsection{Spatial Memory Streaming Prefetcher}
\label{sec:smsPrefetcher}

The Spatial Memory Streaming (SMS) Prefetcher is a spatial locality
prefetcher which tries to exloit access patterns within memory
pages. When designing operating systems and database management
systems the system designer will use memory page as the working unit
to try and limit the amount of bandwidth to the disk. In this way a
datastructure will placed within a page instead of spanning across two
pages. When searching across large and complex datastructures a page
might be accessed in a repitive manner. For example a database search
might always access an index at the start of the page and a pointer
collection at the end of the page for every page it searches trought
while the program runs a in a loop. SMS tries to record such access
patterns and use the same pattern the next time the program is at the
same place, but with a different memory address to be accessed it can
prefetch the recorded access pattern within the page. {\bf (references,
  citations)}

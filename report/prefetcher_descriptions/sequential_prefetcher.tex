\subsection{Sequential Prefetcher}
\label{sec:sequentialPrefetcher}
The sequential prefetcher is based on the fact that memory accesses
often follow a sequential pattern. When accessing the members of an
array, or iterating through a loop, most of the memory accesses will
follow a stride-n pattern (where n is the number of addresses between
the accessed addressess).

The sequential prefetcher keeps track of the last \emph{n} values, and
also keeps track of for how long the current stride has been
ongoing. The prefetchers agressiveness can be configured by
configuring the amount of strides that needs to be discovered before
the prefetcher will react on a memory access.

For each memory access, the memory address is compared to the last
memory address. If the current stride value corresponds to the
difference, the number of current strides is updated. If the number of
current strides is greater than or equal to the required/configured
amunt of strides, a prefetched address is returned.  {\bf (references,
  citations)}
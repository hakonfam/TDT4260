% !TEX root = ../main.tex
\subsection{Markov Prefetcher}
\label{sec:markovPrefetcher}
The markov prefetcher, as described in \cite{Grannas}, is an address based prefetcher. It remembers past sequences of memory misses, and utilizes heuristics on these sequences to predict future memory accesses.

As shown in \cite{Joseph}, it utilizes a directed graph with nodes representing cache blocks, and weighted edges representing the probability that the successor node will be accessed after the predecessor node (see figure \ref{fig:markov}). The weight of the edges is determined by how often the cache block an edge points to is accessed directly after the node the edge is pointing from. For each memory access, the edge corresponding to the current transition needs to be updated. 

When predicting the next sequence of addresses, the markov prefetcher follows the most probable path through the graph. Configuring the markov prefetcher would allow you to decide how many addresses it should prefetch, when to create a new edge between two nodes, when to delete an edge and when to delete a node.

Looking at figure \ref{fig:markov}, if two addresses were to be prefetched, and \emph{0x0001} was the current node, then addresses \emph{0x0002} and \emph{0x0003} would be prefetched.

\begin{figure}[H]
\includegraphics[scale=0.5]{./figures/markov}
\caption{\label{fig:markov}A snippet of a possible weighted graph used by the markov prefetcher.}
\end{figure}


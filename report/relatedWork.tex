% Evaluation criterion:
%- Language and use of figures
%- Clarity of the problem statement
%- Overall document structure
%- Depth of understanding for the field of computer architecture
%- Depth of understanding of the investigated problem

\section{Related Work}
\label{sec:relatedWork}

A large number of different prefetching strategies have been proposed,
with the intent of either being better at adapting to the needs of
different applications or recognizing an access pattern which is not
adequately handled in existing prefetchers \textbf{(citation
  needed)}. The main inspiration for our prefetcher implementations is
Grannæs' paper~\cite{Grannas}. In the background section of this
paper, Grannæs describes five different categories of hardware
prefetchers: sequential prefetchers, instruction based prefetchers,
address based prefetchers, spatial locality prefetchers and linked
data prefetchers. Several examples of the different types of
prefetchers are given, with references to articles discussing some of
the different kinds in more detail.  {\bf (references, citations)}
